%==============================================================================
\chapter{Detector}
\label{sec:det}
%==============================================================================

In this chapter, a detailed description of the proposed PINGU neutrino
telescope and its predecessor IceCube will be given. We will start with
introducing the concept of ice or water based neutrino telescopes based on the
detection of Cherenkov radiation with IceCube/DeepCore as example, followed by
a characterisation of its upcoming PINGU upgrade.
Thereafter we will discuss how physics events will be selected and
reconstructed in an analysis targeting the determination of the neutrino mass
hierarchy, and finally how conceptually new hardware might improve the results.

%==============================================================================
\section{IceCube/DeepCore}
\label{sec:ICDC}
%==============================================================================

As already mentioned (Sec.~\ref{sec:NuDetection}), the natural choice for
observing the low natural fluxes of high energy neutrinos are water-based
Cherenkov detectors. Although the basic requirement---a sufficiently large
amount of water or ice---seems not very difficult to meet, there are additional
constraints that have to be addressed as well:

\begin{description}
 \item[Size:] Depending on the energy range one is interested in, the size of
  the detector has to be adjusted accordingly. Since the atmospheric flux
  decreases rapidly with increasing energy, one needs larger detectors to study
  higher fluxes. Roughly from the GeV scale upwards, the required dimensions are
  so big (several hundred metres) that artificial structures like the
  underground caverns of Kamiokande and Super-Kamiokande \cite{SuperKosc} are
  not feasible any more and one has to look for suitable natural locations.
 \item[Transparency:] Since the detection of neutrinos is based on recording
  Cherenkov radiation, i.\,e.\ photons in the optical and near UV regime,
  obviously the chosen medium has to be transparent for these photons. Here ice
  has an advantage over fluid water as it has very low absorption down to
  wavelengths of 300\,nm and below \cite{IceProps}, while the fluid starts to
  absorb significantly below 400\,nm \cite{WaterAbs}.
 \item[Purity:] Usually the experiments try to reconstruct the neutrino events
  as accurately as possible. Therefore it is desirable to record a large number
  of unscattered photons and hence a very clear environment\footnote{There might
  be, however, situations where scattering is desired, e.\,g.\ when only the
  neutrino energy is of interest, then strong scattering keeps the photons
  inside the detector for a longer time and hence increases the total number
  of detected photons, thereby improving the energy resolution}.
 \item[Shielding:] In high energy neutrino experiments, muons from atmospheric
  showers created by cosmic radiation (cf.\ Sec.~\ref{sec:AtmNus}) are a
  background process whose rate is several orders of magnitude higher than the
  neutrino signal. In order to suppress those muons, detectors have to be
  placed deep underground so that there is a shielding with several hundred
  meters thickness, comparable to the range of $\mathcal{O}
  (100\,\mathrm{GeV})$ muons.
\end{description}



%==============================================================================
\section{PINGU}
\label{sec:PINGU}
%==============================================================================


%==============================================================================
\section{Event Selection}
\label{sec:EvtSel}
%==============================================================================


%==============================================================================
\section{Event Reconstruction}
\label{sec:EvtReco}
% TODO: maybe move this section to the end of the chapter?
%==============================================================================


%==============================================================================
\section{Next-Generation Optical Modules}
\label{sec:EvtReco}
%==============================================================================

\subsection{Wavelength-shifting Optical Module (WOM)}


\subsection{Multi-PMT Optical Module (mDOM)}