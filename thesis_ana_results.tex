%==============================================================================
\section{Results for the Baseline Geometry}
\label{sec:results_baseline}
%==============================================================================

\begin{table}[htpb]
 \caption{Uncertainties on all systematic parameters for the baseline
  detector model with three years of lifetime, ranked according to their impact
  on the mass hierarchy parameter $h$.}
 \label{tab:baseline_results}
 \begin{center}
  \small{\begin{tabular}{lrrrrrr} 
\toprule
Parameter & Impact [\%] & Best Fit & $\sigma^\mathrm{full}$ & $\sigma^\mathrm{stat}$ & $\sigma^\mathrm{syst}$ & Prior \\ 
\midrule
$h$ & 100.0 & \num{1.00e+00} & \num{3.43e-01} & \num{2.33e-01} & \num{2.52e-01} & free \\
$r_{A_\mathrm{eff},\,\nu-\bar\nu}$ & 8.9 & \num{0.00e+00} & \num{4.73e-02} & \num{4.76e-03} & \num{1.47e-01} & \num{5.00e-02} \\
$\vartheta_{13}$ [$^\circ$] & 8.0 & \num{8.93e+00} & \num{4.64e-01} & \num{8.24e-01} & \num{3.67e+00} & \num{4.68e-01} \\
$n_{A_\mathrm{eff}}$ & 7.4 & \num{0.00e+00} & \num{1.94e-02} & \num{1.99e-03} & \num{1.94e-02} & \num{2.00e-01} \\
$\vartheta_{23}$ [$^\circ$] & 3.2 & \num{3.86e+01} & \num{4.67e-01} & \num{3.02e-01} & \num{3.97e-01} & \num{1.32e+00} \\
$\Delta m^2_{31}$ [eV$^2$] & 2.7 & \num{2.46e-03} & \num{6.49e-05} & \num{1.77e-05} & \num{1.10e-04} & \num{8.00e-05} \\
$r_{\Phi,\,\nu_e-\nu_\mu}$ & 2.4 & \num{0.00e+00} & \num{1.10e-02} & \num{5.12e-03} & \num{1.01e-02} & \num{5.00e-02} \\
$s_{A_\mathrm{eff}}$ [m$^2$/GeV] & 1.1 & \num{0.00e+00} & \num{2.19e-04} & \num{1.23e-04} & \num{1.82e-04} & free \\
$\Delta_\mathrm{PID}$ [GeV] & 0.9 & \num{0.00e+00} & \num{1.70e-02} & \num{1.58e-02} & \num{6.23e-03} & \num{5.00e-01} \\
$s_E$ & 0.1 & \num{1.00e+00} & \num{2.81e-02} & \num{7.66e-03} & \num{3.31e-02} & \num{5.00e-02} \\
\bottomrule 
\end{tabular}
}
 \end{center}
\end{table}

Using the settings described in the previous section, the Fisher matrix for
PINGU can now be constructed with \papa. The full, statistical, and systematic
errors are for all parameters are listed in Tab.~\ref{tab:baseline_results} for
a nominal PINGU lifetime of three years. The parameters are ordered after their
\emph{impact} on the mass hierarchy parameter $h$, which is defined as the
square of their correlation coefficient $c_{ih}$ (\ref{eqn:corr_coeff}) with
$h$. Note that for the baseline settings, the systematic parameter
$s_\mathrm{PID}$ has been excluded as it is fully degenerate with $n_{\aeff}$:
since the PID decision is binary, no channel of unidentified events exists and
hence both parameters just evenly increase the overall number of events,
effectively.

From the first line of Tab.~\ref{tab:baseline_results}, one can read off the
expected significance of PINGU's mass hierarchy measurement by inverting the
full error (see Sec.~\ref{sec:fisher_hierarchy}). This gives an expected
significance of 2.9\,$\sigma$ after three years. Looking at the statistical
error alone, this value increases to 4.3\,$\sigma$, emphasising the important
role systematic parameters are playing in the determination of the mass
hierarchy.

Treating the track and cascade channels separately, the expected significances
are 1.9\,$\sigma$ with and 3.0\,$\sigma$ without systematics for the cascade
and 1.3\,$\sigma$ (3.1\,$\sigma$) for the track channel,
respectively\footnote{The full error listings corresponding to
Tab.~\ref{tab:baseline_results} can be found in
App.~\ref{app:fisher_baseline}}. Although the purely statistical significances
are comparable, when taking systematics into account the significance for the
cascades remains much higher than the track significance. 


\subsection{Impact of the Octant of \thet{23}}
\label{sec:results_octant}

\subsection{Measuring the atmospheric mixing parameters}
\label{sec:results_atmosperic}

\subsection{The Missing Monte Carlo Effect}
\label{sec:results_mcstats}

\subsection{High-Purity Event Classification}
\label{sec:results_includeunkn}