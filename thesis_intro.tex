%==============================================================================
\chapter{Introduction}
\label{sec:intro}
%==============================================================================

Although the first conclusive observation of neutrino oscillations was made not
even twenty years ago, this phenomenon of neutrinos changing their flavour when
travelling macroscopic distances has been one of the major areas of research in
particle physics and astrophysics ever since. Up to now, it is the only
manifestation of so-called ``physics beyond the standard model'' that has been
confirmed experimentally. During the past two decades, many dedicated
experiments have mapped out the parameters characterising neutrino oscillations
in great detail, leaving only two parameters to be determined.

One of these parameters is the so-called neutrino mass hierarchy. It refers to
the sign of another parameter, one of the two independent mass splittings,
whose absolute value has already been measured. The fact that the absolutes of
parameters can be determined precisely without learning about its sign is one
of the peculiarities of the neutrino oscillation formalism, where central
parameters enter quadratically in most cases.

A chance to access the neutrino mass hierarchy is to study the differences in
the oscillation probabilities of neutrinos and antineutrinos at low GeV
energies that are created in the Earth's atmosphere and propagate through its
interior. The proposed Precision IceCube Next Generation Upgrade (PINGU) will be
a facility apt to observe the small modulations on top of the flux of
atmospheric neutrinos with the required precision.

As its name suggests, PINGU is planned as an upgrade to the existing IceCube
neutrino telescope at the geographic South Pole in Antarctica. IceCube has been
constructed to discover extra-terrestrial neutrinos at TeV to PeV energies.
Neutrino oscillation patterns in the atmospheric flux at medium GeV energies,
however, have already been observed as well using its DeepCore extension. PINGU
is now intended to further lower the energy threshold down to a regime where
signatures of the neutrino mass hierarchy appear. This also provides an
opportunity to measure the absolute values of the relevant oscillation
parameters with high precision.

In this thesis, the development of an effective detector simulation for PINGU,
named \papa for ``Parametrised PINGU Analysis'', is described. The outcome of
this simulation is analysed using the Fisher Matrix formalism, a tool that is
well established in cosmology, but novel to be applied to a particle physics
experiment. In a linear approximation, it allows for a fast construction of the
full covariance matrix of the experiment including a large number of systematic
uncertainties.

After checking that the prerequisites for the Fisher Matrix are in fact
fulfilled, PINGU's expected sensitivity to the mass hierarchy is evaluated,
showing its dependence on controlling the relevant systematics. The expected
precision in measuring the accessible oscillation parameters is calculated as
well. Following this, the effect of changing various simulation input
parameters is studied in terms of the resulting sensitivity to the neutrino
mass hierarchy.

Finally, \papa is modified to simulate the outcome of JUNO, a nuclear reactor
based neutrino experiment targeted at the determination of the neutrino mass
hierarchy as well, yet exploiting a very different physical effect. JUNO's
sensitivity to the mass hierarchy will be calculated, where a Fourier
transformation of the observed signal is needed in order to justify the
application of the Fisher Matrix. Afterwards, the standalone results of PINGU
and JUNO can be combined easily as in the Fisher Matrix formalism this
corresponds to a mere matrix addition and inversion. Then the benefits from
marginalising over common systematics can be investigated.

The thesis concludes with a summary of the results found using \papa and
analysing its outcome in terms of the Fisher Matrix. An outlook is given on the
future of \papa and its integration into a wider software framework for PINGU
detector simulations.