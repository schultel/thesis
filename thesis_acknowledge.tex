%------------------------------------------------------------------------------
\chapter*{Danksagung}
\label{sec:ack}
%------------------------------------------------------------------------------

\begin{otherlanguage}{ngerman}
Und das war es jetzt\dots\ Fehlt nur noch ein ganz gro\ss{}es Dankesch\"on an 
alle, ohne die diese Arbeit so nicht m\"oglich und die Zeit nicht so sch\"on 
gewesen w\"are.

Zuerst nat\"urlich an Marek Kowalski, der mir schon fr\"uh die Promotionsstelle 
zugesagt und mir sogar -- lang, lang ists her -- zu meiner Diplomarbeit schon 
ein paar n\"utzliche Tips gegeben hat. Auch sp\"ater war trotz vollen 
Terminplans und am Schluss auch gro\ss{}er Entfernung (fast) immer Zeit f\"ur 
ein paar hilfreiche Worte.
Danke auch an Norbert Wermes f\"ur die \"Ubernahme der Zweitkorrektur.

Dann der Bonner Doppel-Arbeitskreis. Sehr spannende Konstruktion, zwei doch 
ziemlich unterschiedliche Themen unter einem Dach. Erst mal vielen Dank an alle 
f\"ur die sch\"one Arbeitsatmosph\"are, den vielen leckeren Kuchen und die 
allmitt\"agliche Tee- und Kaffeepause. Besonderen Dank an meine flei\ss{}igen 
Korrekturleser Andreas, Markus, Alex und Marcel. Und weil Ruth auch was gelesen 
hat, sind meine Erg\"usse hoffentlich auch f\"ur Nicht-IceCuber verst\"andlich. 

Sebastian hat zwar nichts gelesen, aber trotzdem wohl am meisten beigetragen. 
Ob beim Programmieren, Debuggen, bei der Analyse oder bei Pr\"asentationen: 
Immer gab es hilfreiche Kommentare und Anregungen. Unter denen mag ich manchmal 
auch etwas gest\"ohnt haben, aber ein Faultier wie ich braucht den 
gelegentlichen Tritt in den Hintern. Und zur Belohnung ist er ja jetzt Prof 
im goldische Meenz.

Womit wir auch schon bei der Family w\"aren. Allen voran meine Schwestern 
Judith und Johanna, die zwar mittlerweile schon richtig erwachsen, aber 
trotzdem noch die lustigsten und liebsten Schwestern sind, die man haben kann. 
Au\ss{}erdem k\"onnen sie mich vermutlich beide unter den Tisch trinken -- 
Studenten halt. Dazu mein Onkel Mani, der tats\"achlich fast gleichzeitig mit 
mir aus Mainz gen Norden gezogen ist, allerdings nicht ganz so weit. Und meine 
Eltern: Danke f\"ur alles, trotz Allem.

Und zum Schluss noch einmal nach Bonn, zu den Polyphonikern: Ein toller Chor, 
nicht nur musikalisch, mit dem ich mich hier direkt heimisch gef\"uhlt habe. 
Dort hinzugehen, war eine der, wenn nicht die beste und wichtigste Entscheidung 
meines Lebens. 

Denn jetzt habe ich die beste Michi der Welt. Du hast mir das verr\"uckteste 
und sch\"onste Jahr meines Lebens geschenkt, dem noch bessere folgen werden. 
Und den besten Grund \"uberhaupt, endlich mal fertig zu werden. 

Das bin ich hiermit.


\end{otherlanguage}

%%% Local Variables: 
%%% mode: latex
%%% TeX-master: "../mythesis"
%%% End: 
