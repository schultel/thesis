\chapter{Applicability of the Fisher Matrix}
\label{app:fisher_valid}

\chapter{Oscillation Probabilities}
\label{app:oscillation}

\begin{figure}[h]
 \centering
 \includegraphics[width=0.95\textwidth]{osc_ds_nue_to_nue}
 \caption{Oscillation probabilities for $\nue \to \nue$ (top) and $\nuebar \to
          \nuebar$ (bottom) for normal and inverted hierarchy.}
\end{figure}


\begin{figure}[t!]
 \centering
 \includegraphics[width=0.95\textwidth]{osc_ds_nue_to_numu}
 \caption{Oscillation probabilities for $\nue \to \numu$ (top) and $\nuebar \to
          \numubar$ (bottom) for normal and inverted hierarchy.}
\end{figure}

\begin{figure}[b!]
 \centering
 \includegraphics[width=0.95\textwidth]{osc_ds_nue_to_nutau}
 \caption{Oscillation probabilities for $\nue \to \nutau$ (top) and $\nuebar \to
          \nutaubar$ (bottom) for normal and inverted hierarchy.}
\end{figure}

\begin{figure}[t!]
 \centering
 \includegraphics[width=0.95\textwidth]{osc_ds_numu_to_nue}
 \caption{Oscillation probabilities for $\numu \to \nue$ (top) and $\numubar \to
          \nuebar$ (bottom) for normal and inverted hierarchy.}
\end{figure}

\begin{figure}[b!]
 \centering
 \includegraphics[width=0.95\textwidth]{osc_ds_numu_to_numu}
 \caption{Oscillation probabilities for $\numu \to \numu$ (top) and $\numubar
          \to \numubar$ (bottom) for normal and inverted hierarchy.}
\end{figure}


\begin{figure}[t!]
 \centering
 \includegraphics[width=0.95\textwidth]{osc_ds_numu_to_nutau}
 \caption{Oscillation probabilities for $\numu \to \nutau$ (top) and $\numubar
          \to \nutaubar$ (bottom) for normal and inverted hierarchy.}
\end{figure}

\chapter{Parametrisations of the Detector Resolutions}
\label{app:reco_params}

The full parametrisations of the reconstruction performances will be listed in
form of the actual \papa input. These are nested dictionaries giving the
resolutions in energy (\texttt{'e'}) and \coszen (\texttt{'coszen'}) for all
four interaction channels: \nue, \numu, and \nutau CC (\texttt{'nue'},
\texttt{'numu'}, and \texttt{'nutau'}, respectively) and \nux NC
(\texttt{'NC'}). For each of those the resolution is given by the five
parameters \texttt{'fraction'}, \texttt{'loc1'}, \texttt{'loc2'},
\texttt{'width1'}, and \texttt{'width2'}, corresponding to $f$, $\mu_1$,
$\mu_2$, $\sigma_1$ and $\sigma_2$ in (\ref{eqn:reco_param}).

The actual function definitions for the five parameters is then supplied as a
text string that can be interpreted as a python function by python's
\texttt{eval()} function. In those definitions, \texttt{'n'} is a shorthand
for the numpy library \cite{numpy} used for most of the numerical operations in
\papa.

\section{Baseline Settings}

\VerbatimInput{figs/reco_param/reco_default.json}

% \section{Energy Resolution}
% 
% \section{\coszen Resolution}


\chapter{PID Functions}
\label{app:pid}

\section{Baseline Settings}

The particle identification is a binary decision, thus only the track
identification probabilities $P_{\mathrm{channel} \to \mathrm{track}}$ are
listed:
\begin{eqnarray}
 P_{\nue\to\mathrm{track}}(E) =
   0.192 \gauss{\log_{10}(E[\mathrm{GeV}])}{0.878}{0.404} + 0.0309 \\
 P_{\numu\to\mathrm{track}}(E) =
   \frac{0.687}{\stepfunc{\log_{10}(E[\mathrm{GeV}])}{0.683}{0.183}} + 0.0585 \\
 P_{\nutau\to\mathrm{track}}(E) =
   0.197 \gauss{\log_{10}(E[\mathrm{GeV}])}{1.28}{0.466} + 0.0732 \\
 P_{\nux\,\mathrm{NC}\to\mathrm{track}}(E) =
   0.171 \gauss{\log_{10}(E[\mathrm{GeV}])}{1.37}{0.483} + 0.0339
\end{eqnarray}
The cascade identification probabilities are given by $P_{\mathrm{channel} \to
\mathrm{cascade}} = 1 - P_{\mathrm{channel} \to \mathrm{track}}$.


\chapter{Full Error Listings}
\label{app:fisher_output}

Here the full error lists, similar to Tab.~\ref{tab:baseline_results}, for
various detector settings and sub-channels will be collected. These have
not been included in the main text for the sake of better readability. For a
description how to read the tables, refer to the explanations given for
Tab.~\ref{tab:baseline_results} in Sec.~\ref{sec:results_baseline}.

\section{Baseline Settings}
\label{app:fisher_baseline}

\begin{table}[htpb]
 \caption{Same as Tab.~\ref{tab:baseline_results}, but for the cascade channel
  only}
 \begin{center}
  \small{\begin{tabular}{lrrrrrr} 
\toprule
Parameter & Impact [\%] & Best Fit & $\sigma^\mathrm{full}$ & $\sigma^\mathrm{stat}$ & $\sigma^\mathrm{syst}$ & Prior \\ 
\midrule
$h$ & 100.0 & \num{1.00e+00} & \num{5.17e-01} & \num{3.39e-01} & \num{3.91e-01} & free \\
$r_{\Phi,\,\nu_e-\nu_\mu}$ & 24.9 & \num{0.00e+00} & \num{2.44e-02} & \num{1.06e-02} & \num{2.58e-02} & \num{5.00e-02} \\
$\vartheta_{23}$ [$^\circ$] & 20.7 & \num{3.86e+01} & \num{1.05e+00} & \num{5.78e-01} & \num{1.61e+00} & \num{1.32e+00} \\
$\Delta_\mathrm{PID}$ [GeV] & 11.4 & \num{0.00e+00} & \num{1.53e-01} & \num{4.14e-02} & \num{1.56e-01} & \num{5.00e-01} \\
$s_{A_\mathrm{eff}}$ [m$^2$/GeV] & 4.5 & \num{0.00e+00} & \num{3.96e-04} & \num{1.85e-04} & \num{3.50e-04} & free \\
$r_{A_\mathrm{eff},\,\nu-\bar\nu}$ & 4.2 & \num{0.00e+00} & \num{4.93e-02} & \num{5.55e-03} & \num{3.06e-01} & \num{5.00e-02} \\
$\vartheta_{13}$ [$^\circ$] & 3.0 & \num{8.93e+00} & \num{4.66e-01} & \num{1.67e+00} & \num{5.53e+00} & \num{4.68e-01} \\
$s_E$ & 1.5 & \num{1.00e+00} & \num{3.14e-02} & \num{1.92e-02} & \num{3.55e-02} & \num{5.00e-02} \\
$\Delta m^2_{31}$ [eV$^2$] & 1.2 & \num{2.46e-03} & \num{6.73e-05} & \num{4.44e-05} & \num{1.16e-04} & \num{8.00e-05} \\
$s_\mathrm{PID}$ & 0.1 & \num{1.00e+00} & \num{2.31e-02} & \num{2.32e-03} & \num{2.29e-02} & free \\
\bottomrule 
\end{tabular}
}
 \end{center}
\end{table}

\begin{table}[htpb]
 \caption{Same as Tab.~\ref{tab:baseline_results}, but for the track channel
  only}
 \begin{center}
  \small{\begin{tabular}{lrrrrrr} 
\toprule
Parameter & Impact [\%] & Best Fit & $\sigma^\mathrm{full}$ & $\sigma^\mathrm{stat}$ & $\sigma^\mathrm{syst}$ & Prior \\ 
\midrule
$h$ & 100.0 & \num{1.00e+00} & \num{7.47e-01} & \num{3.20e-01} & \num{6.75e-01} & free \\
$s_E$ & 8.4 & \num{1.00e+00} & \num{2.98e-02} & \num{8.36e-03} & \num{3.62e-02} & \num{5.00e-02} \\
$\Delta m^2_{31}$ [eV$^2$] & 7.2 & \num{2.46e-03} & \num{6.70e-05} & \num{1.93e-05} & \num{1.21e-04} & \num{8.00e-05} \\
$\vartheta_{23}$ [$^\circ$] & 6.0 & \num{3.86e+01} & \num{5.83e-01} & \num{3.55e-01} & \num{5.43e-01} & \num{1.32e+00} \\
$\vartheta_{13}$ [$^\circ$] & 3.6 & \num{8.93e+00} & \num{4.67e-01} & \num{9.47e-01} & \num{1.01e+01} & \num{4.68e-01} \\
$r_{\Phi,\,\nu_e-\nu_\mu}$ & 2.4 & \num{0.00e+00} & \num{3.59e-02} & \num{5.84e-03} & \num{5.12e-02} & \num{5.00e-02} \\
$\Delta_\mathrm{PID}$ [GeV] & 2.3 & \num{0.00e+00} & \num{3.80e-02} & \num{1.71e-02} & \num{3.41e-02} & \num{5.00e-01} \\
$n_{A_\mathrm{eff}}$ & 1.3 & \num{0.00e+00} & \num{3.00e-02} & \num{3.87e-03} & \num{3.01e-02} & \num{2.00e-01} \\
$s_{A_\mathrm{eff}}$ [m$^2$/GeV] & 0.3 & \num{0.00e+00} & \num{4.13e-04} & \num{1.64e-04} & \num{3.79e-04} & free \\
$r_{A_\mathrm{eff},\,\nu-\bar\nu}$ & 0.2 & \num{0.00e+00} & \num{4.86e-02} & \num{9.29e-03} & \num{2.06e-01} & \num{5.00e-02} \\
\bottomrule 
\end{tabular}
}
 \end{center}
\end{table}
