%==============================================================================
\chapter{Conclusion}
\label{sec:conclusion}
%==============================================================================

In this thesis, the development of \papa, a standalone simulation for the
planned PINGU atmospheric neutrino experiment, has been described. With \papa,
the expected event histograms in ($E$,\,\coszen) can be generated directly from
inputs like the atmospheric flux, effective areas, oscillation probabilities
and a parametrisation of the detector resolution, taking into account a variety
of systematic parameters, cf.\ Sec.~\ref{sec:papa}. Since most of the inputs are
extracted from Monte Carlo data in advance, the time-consuming event-by-event
simulation can be avoided in \papa, making it well suited to explore PINGU's
multi-dimensional parameter space in short time and robust even in case of
rather low Monte Carlo statistics.

Using the results of \papa, PINGU's capability to measure the oscillation
parameters \thet{23} and \dm{31} and especially to determine the neutrino mass
hierarchy, \ie the sign of \dm{31}, was assessed with the Fisher matrix
technique. This analysis method, whose application to a particle physics
experiment is novel, is especially suited for problems with a large number of
parameters and essentially linear detector response to these parameters.

After checking that the requirements for applying the Fisher matrix are met
(Sec.~\ref{sec:fisher_prereq}), PINGU's performance was evaluated for its
current baseline geometry V36 with the most up-to-date estimates of
reconstruction resolution. In Sec.~\ref{sec:results_baseline} it was shown
that for a nominal lifetime of three years, PINGU is expected to determine the
neutrino mass hierarchy with a confidence level of 2.9\,$\sigma$, with the major
contribution coming from the analysis of cascade-like events\footnote{All events
that are not caused by a \numu or \numubar CC interaction, \ie without an
outgoing muon.}. In addition, PINGU will provide precision measurements of the
oscillation parameters \dm{31} and \thet{23}, and in particular resolve the
octant of \thet{23}. The latter is possible as PINGU observes neutrino
oscillations in matter, where the symmetry between the two octants is broken.
This also means that the asymmetry between the normal and inverted mass
hierarchy cases grows with an increasing value of \thet{23}, such that the
expected significance of the mass hierarchy measurement can reach 5\,$\sigma$
and more if the true value of \thet{23} is in the second octant.
The use of next-generation optical sensors---like the WOM concept, for which
basic work has been done in the context of this thesis---can further enhance
PINGU's performance by increasing the expected photon statistics per event and
hence improving the detection threshold and reconstruction precision
(Sec.~\ref{sec:om_effects}).

Finally, in Sec.~\ref{sec:JUNO} a combined analysis of PINGU and JUNO, a medium
baseline reactor neutrino experiment currently under construction, was done to
explore possible synergies. Given the respective settings, JUNO's expected
neutrino spectrum could be simulated with \papa without any modifications. Yet
for the analysis with the Fisher matrix method the spectrum had to be Fourier
transformed to achieve linear relations between the observables and the physics
parameters. After this, JUNO's sensitivity to the neutrino mass hierarchy as
reported by the JUNO collaboration itself could be reproduced. The combined
analysis of both experiments, \ie adding the respective Fisher matrices,
results in a NMH significance exceeding the squared sum of the individual ones
by 0.5\,$\sigma$, mainly due to PINGU profiting from JUNO's constraint on the
oscillation parameter \thet{13}\footnote{For this study, no detector systematics
were considered and no external constraints were put on the oscillation
parameters.}.

In a broader picture, intermediate results of this work have been published as
official sensitivity estimates by the PINGU collaboration. The current version
of the PINGU Letter of Intent \cite{LoI} relies on the \papa simulation of the
previous baseline geometry V15, evaluated with the Fisher matrix technique. An
update of this report is planned in the near future, referring to the
current baseline geometry V36, reflecting all improvements on event
selection, reconstruction etc.\ achieved in the meantime, and covering a wider
range of systematic parameters.

In this update, the analysis will only partly depend on \papa. The main
calculations will be done with \texttt{pisa} \cite{pisa}, the software framework
\papa has evolved into. It keeps the idea of a direct simulation of event
histograms in a staged way and develops it further. Separating the individual
stages even more than \papa, \texttt{pisa} offers the option to add alternative
implementations of particular stages\footnote{E.\,g.\ other oscillation codes,
event reconstruction from parametrised reconstruction functions or directly
from Monte Carlo data, \dots} and easily apply different analysis
methods\footnote{E.\,g.\ the Fisher Matrix and the Likelihood Ratio techniques.}
to exactly the same data, as well as better maintainability of the code itself.

Since \papa and the Fisher Matrix analysis have been fully re-implemented
within \texttt{pisa} and both input and output format are compatible, the
results presented in this thesis can always be reproduced. Vice versa, the
concepts first introduced here, namely the staged simulation of event
histograms using parametrisations fitted to distributions of Monte Carlo events
and the Fisher matrix analysis, leading to a very fast evaluation of systematic
effects, will stay present within PINGU and might even be adopted by other
collaborations.