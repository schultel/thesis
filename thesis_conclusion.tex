%==============================================================================
\chapter{Conclusion}
\label{sec:conclusion}
%==============================================================================

In this thesis, the development of \papa, a standalone simulation for the
planned PINGU atmospheric neutrino experiment, has been described. With \papa,
the expected event histograms in ($E$,\,\coszen) can be generated directly from
inputs like the atmospheric flux, effective areas, oscillation probabilities
and a parametrisation of the detector resolution, taking into account a variety
of systematic parameters, cf.\ Sec.~\ref{sec:papa}. Since most of the inputs are
extracted from Monte Carlo data in advance, the time-consuming event-by-event
simulation can be avoided in \papa, making it well suited to explore PINGU's
multi-dimensional parameter space in short time and robust even in case of
rather low Monte Carlo statistics.

Using the results of \papa, PINGU's capability to measure the oscillation
parameters \thet{23} and \dm{31} and especially to determine the neutrino mass
hierarchy, \ie the sign of \dm{31}, was assessed with the Fisher matrix
technique. This analysis method, whose application to a particle physics
experiment is novel, is especially suited for problems with a large number of
parameters and essentially linear detector response to these parameters.

After checking that the requirements for applying the Fisher matrix are met
(Sec.~\ref{sec:fisher_prereq}), PINGU's performance was evaluated for its
current baseline geometry V36 with the most up-to-date estimates of
reconstruction resolution. In Sec.~\ref{sec:results_baseline} it was shown
that for a nominal lifetime of three years, PINGU is expected to determine the
neutrino mass hierarchy with a confidence level of 2.9\,$\sigma$, with the major
contribution coming from the analysis of cascade-like events\footnote{All events
that are not caused by a \numu or \numubar CC interaction, \ie without an
outgoing muon.}. In addition, PINGU will provide precision measurements of the
oscillation parameters \dm{31} and \thet{23}, and in particular resolve the
octant of \thet{23}. The latter is possible as PINGU observes neutrino
oscillations in matter, where the symmetry between the two octants is broken.
This also means that the asymmetry between the normal and inverted mass
hierarchy cases grows with an increasing value of \thet{23}, such that the
expected significance of the mass hierarchy measurement can reach 5\,$\sigma$
and more if the true value of \thet{23} is in the second octant.