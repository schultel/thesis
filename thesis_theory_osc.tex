%==============================================================================
\section{Neutrino Oscillations}
\label{sec:osc}
%==============================================================================

The Standard Model of Particle Physics, as described in the previous section,
has been one of the most successful theories in the history of physics. It is,
however, not fundamental in the sense that it can explain all physical
phenomena alone. Its shortcomings are e.\,g.\ the missing inclusion of the
fourth fundamental force, gravity, and an explanation of the fundamental
asymmetry between bosons and fermions.
There are theoretical extensions to the Standard Model addressing these
questions, such as quantum gravity \cite{QuantumGravity} and supersymmetry
\cite{SUSY}, but all of them lack experimental evidence so far.

Yet there is one effect of so-called ``Physics beyond the Standard Model'' that
has been well established experimentally during the past years: neutrino
oscillations. As already mentioned in Sec.~\ref{sec:NusInSM}, this term refers
to neutrinos changing their flavour when travelling over macroscopic distances,
which can be explained by finite neutrino masses, while in the Standard Model
they have zero mass. The theory behind this process will be described in the
following.

\subsection{Vacuum Oscillations}
\label{sec:VacOsc}

There are two bases of eigenstates to which a neutrino can be decomposed: the
flavour and the mass base. The flavour eigenstates are \ket{\nue}, \ket{\numu},
and \ket{\nutau}, which will be summarised as \ket{\nu_\alpha}. These are the
eigenstates of the weak interaction, hence neutrinos are always produced as a
pure flavour eigenstate and have to be projected back onto these eigenstates
whenever they interact.

On the other hand there are the three mass eigenstates \ket{\nu_1}, \ket{nu_2},
and \ket{\nu_3}, summarised as \ket{\nu_i}, corresponding to the three neutrino
masses $m_i$. The absolute values of these masses are yet unknown, but since
neutrino oscillations have been observed, at least two of them have to be
different from zero. The mass eigenstates have to be considered when describing
the propagation of a neutrino in vacuum since the corresponding Hamiltonian
\begin{equation}
 \hat{H} = -\frac{\hbar^2}{2m}\nabla^2
\end{equation}
only depends on its mass.

Changes between the two bases are carried out via the so-called PMNS
matrix\footnote{After Bruno Pontecorvo, Ziro Maki, Masami Nagakawa, and Shoichi
Sakata}


\subsection{Neutrino Mass Hierarchy}
\label{sec:NMH}


\subsection{Oscillations in Matter}
\label{sec:MSW}


\subsection{Oscillation Experiments}
\label{sec:OscExp}