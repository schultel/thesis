%==============================================================================
\section{Neutrino Oscillations}
\label{sec:osc}
%==============================================================================

The Standard Model of Particle Physics, as described in the previous section,
has been one of the most successful theories in the history of physics. It is,
however, not fundamental in the sense that it can explain all physical
phenomena alone. Its shortcomings are e.\,g.\ the missing inclusion of the
fourth fundamental force, gravity, and an explanation of the fundamental
asymmetry between bosons and fermions.
There are theoretical extensions to the Standard Model addressing these
questions, such as ``Grand Unified Theories'', supersymmetry, and many others 
\cite{Nagashima}, but all of them lack experimental evidence so far.

Yet there is one effect of so-called ``Physics beyond the Standard Model'' that
has been well established experimentally during the past years: neutrino
oscillations. As already mentioned in Sec.~\ref{sec:NusInSM}, this term refers
to neutrinos changing their flavour when travelling over macroscopic distances,
which can be explained by finite neutrino masses, while in the Standard Model
they have zero mass.

The theory behind this process will be described in the following. In-depth
treatments of this topic can be found in many textbooks, e.\,g.\
\cite{GiuntiKim, Zuber, Nagashima, XingZhou}. In terms of notation, we will
follow \cite{GiuntiKim} here.

\subsection{Vacuum Oscillations}
\label{sec:VacOsc}

% TODO: Find textbook as reference

There are two bases of eigenstates to which a neutrino can be decomposed: the
flavour and the mass base. The flavour eigenstates are \ket{\nue}, \ket{\numu},
and \ket{\nutau}, which will be summarised as \ket{\nu_\alpha}. These are the
eigenstates of the weak interaction, hence neutrinos are always produced as a
pure flavour eigenstate and have to be projected back onto these eigenstates
whenever they interact.

On the other hand there are the three mass eigenstates \ket{\nu_1}, \ket{\nu_2},
and \ket{\nu_3}, summarised as \ket{\nu_k}, corresponding to the three neutrino
masses $m_k$. The absolute values of these masses are yet unknown, but since
neutrino oscillations have been observed, at least two of them have to be
different from zero. The mass eigenstates have to be considered when describing
the propagation of a neutrino in vacuum since they are the eigenstates of the
corresponding Hamiltonian
\begin{equation}
%  \hat{H} = -\frac{\hbar^2}{2m}\nabla^2
 \hat{H}\,\ket{\nu_k} = E_k\,\ket{\nu_k} \quad.
 \label{eqn:vac_hamiltonian}
\end{equation}
% only depends on its mass.

\subsubsection{General Case}

Changes between the two bases are carried out via the so-called PMNS
matrix\footnote{After Bruno Pontecorvo, Ziro Maki, Masami Nagakawa, and Shoichi
Sakata.} $\mathcal{U}_\mathrm{PMNS}$ that can be parametrised using three Euler
angles \thet{ij}, also called mixing angles, and one complex phase angle
$\delta$ that is related to possible CP violation:
\begin{equation}
 \mathcal{U}_\mathrm{PMNS} =
 \begin{pmatrix}
  U_{e1} & U_{e2} & U_{e3} \\
  U_{\mu 1} & U_{\mu 2} & U_{\mu 3} \\
  U_{\tau 1} & U_{\tau 2} & U_{\tau 3}
 \end{pmatrix}
 =
 \begin{pmatrix}
  1 & 0 & 0 \\
  0 & c_{23} & s_{23} \\
  0 & -s_{23} & c_{23}
 \end{pmatrix}
 \begin{pmatrix}
  c_{13} & 0 & s_{13}e^{-i\delta} \\
  0 & 1 & 0 \\
  -s_{13}e^{i\delta} & 0 & c_{13}
 \end{pmatrix}
 \begin{pmatrix}
  c_{12} & s_{12} & 0 \\
  -s_{12} & c_{12} & 0 \\
  0 & 0 & 1
 \end{pmatrix}
\label{eqn:PMNS}
\end{equation}
Here, $s_{ij}$ and $c_{ij}$ are shorthands for $\sin\thet{ij}$ and
$\cos\thet{ij}$, respectively. Transformations between flavour and mass base
are then given by
\begin{eqnarray}
 \ket{\nu_\alpha} = \sum_i U^*_{\alpha k}\, \ket{\nu_k}\quad, \\
 \ket{\nu_k} = \sum_\alpha U_{\alpha k}\, \ket{\nu_\alpha}\quad.
\end{eqnarray}

To ensure lepton number conservation, unitarity of $\mathcal{U}_\mathrm{PMNS}$
has to be required. Any deviation from this can be interpreted as a hint for
additional neutrino flavours that don't participate in the weak
interaction\footnote{Measurements of the $Z^0$ decay width have shown that only
three weakly interacting neutrino flavours exist \cite{Zwidth}---at least at
masses up to half the $Z^0$ mass, $m_\nu < m_{Z^0}/2 = 45.6$\,GeV.}. Such
signals have been reported (e.\,g.\ \cite{MiniBooNE}), but the overall
picture remains inconclusive \cite{PDG}.

If now a pure flavour eigenstate \ket{\nu_\alpha} is produced at an energy $E$,
the probability to detect it as \ket{\nu_\beta} after propagating over a
distance $L$ has to be calculated according to
\begin{equation}
 P_{\alpha\to\beta} = \left|\Braket{\nu_\beta
                            \left|\,\mathcal{U}^\dagger\,\left|\,\hat{H}
                             \,\right|\,\mathcal{U}\,\right|
                             \nu_\alpha}\right|^2 \quad.
 \label{eqn:osc_prob_hamiltonian}
\end{equation}
In the mass base, the propagation of the neutrinos can be described as plain
waves,
\begin{equation}
 \Ket{\nu_i(t)} = \exp\left(-i(E_i t - \vec{p}_i\cdot \vec{x})\right)\,
  \Ket{\nu_i(0)} \quad,
\end{equation}
and assuming relativistic neutrinos ($m_i \ll E_i \Rightarrow v \approx c$) one
can approximate in natural units\footnote{$\hbar = c = 1$}
\begin{eqnarray}
 E_i =       \sqrt{ \vec{p}_i^2 - m_i^2 }
     \approx p_i + \frac{m_i^2}{2p_i} \quad, \\
 E_i t - \vec{p}_i\cdot \vec{x} = \left(p_i + \frac{m_i^2}{2p_i}\right) L
                                  - p_i L \approx \frac{m_i^2}{2E}
\end{eqnarray}
With this, (\ref{eqn:osc_prob_hamiltonian}) reduces to
\begin{equation}
%  P_{\alpha\to\beta} = \left|\sum_i U^*_{\beta i} U_{\alpha i} \,
%                       e^{-i m_i^2 L / 2E} \right|^2 \quad.
 P_{\alpha\to\beta} = \sum_{k,j} U^*_{\alpha k} U_{\beta k} U_{\alpha j}
                        U^*_{\beta j}
                        \exp\left( -i\frac{\dm{kj}L}{2E} \right)
 \label{eqn:osc_prob_reduced}
\end{equation}
with the squared mass differences
\begin{equation}
 \dm{kj} \equiv m^2_k - m^2_j \quad.
\end{equation}
Here we note that obviously in the mass base the Hamiltonian can be replaced by
an effective one containing the squared masses:
\begin{equation}
 \hat{H}^\mathrm{eff}
 = \frac{1}{2E}\,\mathrm{diag}\left(m_1^2,\,m_2^2,\,m_3^2\right)
 = \frac{m_1^2}{2E}\mathlarger{\mathlarger{\mathbbm{1}}} +
   \frac{1}{2E}\,\mathrm{diag}\left(0,\,\dm{21},\,\dm{31}\right)
 \label{eqn:vac_eff_hamiltonian}
\end{equation}
We can even ignore the first summand on the r.\,h.\,s.\ of the above equation
since it will only introduce an unobservable global phase shift. This
simplification will prove handy when discussing oscillations in matter in
Sec.~\ref{sec:matter_osc}.

Using the unitarity of $\mathcal{U}_\mathrm{PMNS}$, (\ref{eqn:osc_prob_reduced})
can now be rewritten as
\begin{eqnarray}
 P_{\alpha\to\beta} = \delta_{\alpha\beta}
                      &-& 2\sum_{k>j} \Re\left[ U^*_{\alpha k} U_{\beta k}
                                              U_{\alpha j} U^*_{\beta j} \right]
                                    \left[1-\cos \frac{\dm{kj}L}{2E} \right]
                                    \nonumber \\
                      &+& 2\sum_{k>j} \Im\left[ U^*_{\alpha k} U_{\beta k}
                                              U_{\alpha j} U^*_{\beta j} \right]
                                    \sin \frac{\dm{kj}L}{2E} \quad.
\end{eqnarray}

Obviously, this oscillation probability collapses to $P_{\alpha\to\beta} =
\delta_{\alpha\beta}$ if all $m_i$ are equal. Hence the observation of actual
flavour conversion means that the $m_i$ are different from each other and in
particular different from zero (at least two of them), contradicting the
standard model prediction of vanishing neutrino masses. Additionally, the
second oscillatory term only contributes if there actually is CP violation in
the neutrino sector---otherwise the mixing matrix (\ref{eqn:PMNS}) is real.

For antineutrinos, the oscillation probability is derived fully analogue, only
with $\mathcal{U}_\mathrm{PMNS}$ being replaced by its complex conjugate. Hence
differing vacuum oscillation probabilities for neutrinos and antineutrinos are
a proof of CP violation.

\subsubsection{Two Flavour Case}

In many cases\footnote{In particular, if the survival probability of a certain
flavour is measured.} it is sufficient to consider only two neutrino flavours in
the oscillation. Then there is only one mass splitting
\begin{equation}
 \dm{} \equiv \dm{21} \equiv m^2_2 - m^2_1
\end{equation}
and the mixing matrix $\mathcal{U}$ can be parametrised by one effective mixing
angle
\begin{equation}
 \mathcal{U} =
 \begin{pmatrix}
 \cos\vartheta & \sin\vartheta \\
 - \sin\vartheta & \cos\vartheta
 \end{pmatrix} \quad.
\end{equation}
The expression for the transition probability simplifies to
\begin{equation}
 P_{\alpha\to\beta} = \sin^2 2\vartheta \sin^2\left( \frac{\dm{}L}{4E} \right)
                    = \sin^2 2\vartheta\sin^2\left(\pi\frac{L}{L^\mathrm{osc}}
                       \right)\quad,
 \label{eqn:twoflavour_prob}
\end{equation}
introducing the oscillation length
\begin{equation}
 L^\mathrm{osc} \equiv \frac{4\pi\,E}{\dm{}}
  \approx 2.47 \frac{E\,[\si{\GeV}]}{\dm{}\,[\si{\eVsq}]} \si{\km} \qquad
  (\alpha \neq \beta) \quad.
\end{equation}

From (\ref{eqn:twoflavour_prob}), the two different groups of parameters in
neutrino oscillation phenomenology and how they influence the oscillation
probabilities, become obvious:
The mixing angles define the amplitude of the oscillation, with $\vartheta
\approx \ang{45}$ giving rise to so-called ``maximum mixing'' where a full
transition from one flavour to another is possible. The mass splittings
determine the frequency at a given neutrino energy, expressed through the
oscillation length at which the first oscillation minimum appears.

So from an experimental point of view, placing a detector at a distance $L =
L^\mathrm{osc}$ from the neutrino source is preferential, since here the
oscillation effects are strongest. If $L \ll L^\mathrm{osc}$, the flavour
transition has not yet happened while at $L \gg L^\mathrm{osc}$ only the
average transition probability
\begin{equation}
 \left\langle P_{\alpha\to\beta} \right\rangle
  = \frac{1}{2}\sin^2 2\vartheta
\end{equation}
can be measured and no information on \dm{} can be obtained\footnote{Here it
is assumed that in a real experiment one will always have a continuous
neutrino energy spectrum and hence a distribution of oscillation lengths.
Hence the fast oscillations will smear out on a distance $L \gg
L^\mathrm{osc}$.}.

% \subsubsection{Current Status of Neutrino Mixing Parameters}
% \label{sec:MixingParams}

\subsection{Neutrino Mass Hierarchy}
\label{sec:NMH}

Although the existence of neutrino masses is essential for neutrino 
oscillations, the relevant parameters are not the three mass eigenstates 
themselves, but their squared differences \dm{kj}. So they are not three, but 
only two independent parameters since obviously
\begin{equation}
 \dm{31} = \dm{32} + \dm{21} \quad.
 \label{eqn:mass_diff_link}
\end{equation}
On the other hand, apart from CP violating effects, which have not yet been 
observed, all oscillatory terms above are proportional to either $\cos$ or 
$\sin^2$ and hence insensitive to the sign of their argument.

Thus in vacuum oscillations, only information on the distances of the mass 
eigenstates can be collected, but neither on the absolute masses or their 
relative ordering. In other words, studying oscillations one can measure the 
difference between e.\,g.\ the (squared) mass eigenstates $m_1$ and $m_2$, but 
not their actual values and not even whether $m_2$ is larger or smaller than 
$m_1$. This could be resolved, however, if one would measure all three mass
splittings separately and then used (\ref{eqn:mass_diff_link}) to separate the
ordering. Unfortunately, it turns out \cite{Fogli, GonzalezGarcia} that
\begin{equation}
 \dm{32} \simeq \dm{31} \gg \dm{21} \quad,
\end{equation}
such that with current experiments' precision it is impossible to disentangle
\dm{32} and \dm{31}. So what can be done to fix the neutrino mass eigenstates?

To establish absolute neutrino masses one has to consider other effects, such 
as distortions at the upper end of the energy spectrum of nuclear $\beta$ 
decays (as described in Sec.~\ref{sec:BetaDecay}). Here the decay of tritium is 
promising due to its small decay energy. In fact, currently the most stringent 
upper limits for the mass of the electron (anti-) neutrino
\footnote{Defined as $m_{\nue}^2 = \sum_i 
|U^2_{ei}|\, m_i^2$ \cite{NuMassReview}.} set by the Mainz and Troitsk 
experiments \cite{MainzNuMass, TroitskNuMass} stem from this very decay. Their 
limit of 
\begin{equation}
 m_{\nuebar} \lesssim \SI{2.1}{\eV}
\end{equation}
is expected to be improved by one order of magnitude in the KATRIN experiment 
that also targets the tritium decay spectrum \cite{KATRIN}.

However these experiments can only directly measure the superposition of mass
eigenstates that corresponds to the \nue or \nuebar flavour eigenstate. A direct
measurement of the \numu or \nutau mass (or their antiparticles) is by far more
difficult since they cannot be created in a controllable source like a nuclear
decay. The only not completely unrealistic options here would be time-of-flight
measurements with an extremely long baseline\footnote{E.\,g.\ astrophysical
neutrinos that can be associated with a transient event such as a supernova or a
gamma-ray burst.}. Thus oscillations are needed to establish the relative
positions of the mass eigenstates. But, as already mentioned, the ordering or
\emph{hierarchy} of the masses cannot be resolved in oscillations as they were
presented above.

Yet there is a possibility to access the neutrino mass hierarchy in oscillation
experiments, if the neutrinos in question pass through a sufficient amount of
matter along their path. In this case, additional resonances appear that depend
on the sign of the mass splittings. These so-called matter effects will be
discussed in the following section.

\subsection{Oscillations in Matter}
\label{sec:matter_osc}

\begin{figure}
 \centering
 \subfloat[\label{fig:coh_CC}]{
  \begin{fmffile}{coh_CC}
    \begin{fmfgraph*}(40,30) \fmfpen{thin}
      \fmfstraight
      \fmfleft{i00,i0,i2,i3,i41,i42,i5,i51,i6,i1}
      \fmfright{o00,o0,o2,o3,o41,o42,o5,o51,o6,o1}
      \fmf{fermion}{i0,v0,o0}
      \fmflabel{$e^-$}{i0}
      \fmflabel{$\nue$}{o0}
      \fmf{fermion}{i1,v1,o1}
      \fmflabel{$\nue$}{i1}
      \fmflabel{$e^-$}{o1}
      \fmf{dashes,lab=$W$,l.side=left,tension=1.}{v0,v1}
    \end{fmfgraph*}
  \end{fmffile}
  \write18{mpost coh_CC}
 }
 \hspace{2cm}
 \subfloat[\label{fig:coh_NC}]{
  \begin{fmffile}{coh_NC}
    \begin{fmfgraph*}(40,30) \fmfpen{thin}
      \fmfstraight
      \fmfleft{i00,i0,i2,i3,i41,i42,i5,i51,i6,i1}
      \fmfright{o00,o0,o2,o3,o41,o42,o5,o51,o6,o1}
      \fmf{fermion}{i0,v0,o0}
      \fmflabel{$e^-,\,p,\,n$}{i0}
      \fmflabel{$e^-,\,p,\,n$}{o0}
      \fmf{fermion}{i1,v1,o1}
      \fmflabel{$\nu_x$}{i1}
      \fmflabel{$\nu_x$}{o1}
      \fmf{dashes,lab=$Z^0$,l.side=left,tension=1.}{v0,v1}
    \end{fmfgraph*}
  \end{fmffile}
  \write18{mpost coh_NC}
 }
 \caption{Feynman diagrams for the charged \protect\subref{fig:coh_CC} and
  neutral \protect\subref{fig:coh_NC} current contributions to coherent forward
  scattering in matter.}
\label{fig:coherent_scattering}
\end{figure}


When neutrinos are passing through matter, they will undergo coherent forward
scattering off the electrons and nucleons in their path. Since the matter
distribution is continuous, this can be interpreted as a matter potential which
the neutrinos are experiencing, leading to a change of their effective mass
similar to photons passing through a transparent medium.

Neutral current interactions, as shown in Fig.~\ref{fig:coh_NC}, are open to
all neutrino flavours in a similar way. Here the contributions from electrons
and protons cancel since their associated weak currents have opposite
sign\footnote{Assuming that the matter is macroscopically neutral, hence
electrons and protons have equal number densities.} and only the neutron
potential remains:
\begin{equation}
 V_\mathrm{NC} = -\frac{1}{2}\sqrt{2}\,G_F\,N_n
\end{equation}
But since this potential affects all flavours and hence all mass eigenstates in
the same way, it also changes all effective masses by the same amount, but
leaves the mass splittings unaffected.
Charged current scattering (Fig.~\ref{fig:coh_CC}) on the other hand is only
possible for \nue as electrons are the only charged leptons present in ordinary
matter. The CC matter potential can be expressed as
\begin{equation}
 V_\mathrm{CC} = \sqrt{2}\,G_F\,N_e \quad.
\end{equation}

This means that the eigenvalue equation (\ref{eqn:vac_hamiltonian}) has to be
modified to include the matter potential. Since the potential acts on the
flavour eigenstates, we first have to transfer the effective vacuum Hamiltonian
(\ref{eqn:vac_eff_hamiltonian}) into the flavour base and then add the
potential:
\begin{equation}
 \hat{H}_\mathrm{matter} = \mathcal{U}^\dagger \hat{H}_\mathrm{eff}\,\mathcal{U}
                           + \mathrm{diag}\left(V_\mathrm{NC}+V_\mathrm{CC},\,
                                            V_\mathrm{NC},\,V_\mathrm{NC}\right)
\end{equation}
We can now neglect constant contributions again which leaves us with the
following effective Hamiltonian in matter:
\begin{equation}
 \hat{H}_\mathrm{matter}^\mathrm{eff} =
   \frac{1}{2E}\mathcal{U}^\dagger
     \mathrm{diag}\left(0,\,\dm{21},\,\dm{31}\right)\,\mathcal{U}
   + \mathrm{diag}\left(V_\mathrm{CC},\,0,\,0\right)
 \label{eqn:matter_eff_hamiltonian}
\end{equation}


\subsubsection{MSW Effect}
\label{sec:MSW}

\subsubsection{Parametric Resonances}
\label{sec:ParamRes}

\subsection{Oscillation Experiments}
\label{sec:OscExp}

\subsubsection{Solar Neutrinos}

\subsubsection{Atmospheric Neutrinos}

\subsubsection{Reactor Neutrinos}

\subsubsection{Neutrino Beams}

