%==============================================================================
\section{Neutrino Oscillations}
\label{sec:osc}
%==============================================================================

The Standard Model of Particle Physics, as described in the previous section,
has been one of the most successful theories in the history of physics. It is,
however, not fundamental in the sense that it can explain all physical
phenomena alone. Its shortcomings are e.\,g.\ the missing inclusion of the
fourth fundamental force, gravity, and an explanation of the fundamental
asymmetry between bosons and fermions.
There are theoretical extensions to the Standard Model addressing these
questions, such as quantum gravity\cite{QuantumGravity} and
supersymmetry\cite{SUSY}, but all of them lack experimental evidence so far.

Yet there is one effect of so-called ``Physics beyond the Standard Model'' that
has been well established experimentally during the past years: neutrino
oscillations. As already mentioned in Sec.~\ref{sec:NusInSM}, this term refers
to neutrinos changing their flavour when travelling over macroscopic distances,
which can be explained by finite neutrino masses, while in the Standard Model
they have zero mass. The theory behind this process will be described in the
following.

\subsection{Vacuum Oscillations}
\label{sec:VacOsc}


\subsection{Neutrino Mass Hierarchy}
\label{sec:NMH}


\subsection{Oscillations in Matter}
\label{sec:MSW}


\subsection{Oscillation Experiments}
\label{sec:OscExp}