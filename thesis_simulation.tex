%==============================================================================
\chapter{Simulation}
\label{sec:sim}
%==============================================================================

As this thesis is about estimating the performance of the PINGU detector before
it is actually built, the estimate has to be based on simulations. In the
chapter on hand, the simulations used to generate the results reported in the
following chapter will be described in detail.

The chapter, as the simulation process, is divided into two sections.
Sec.~\ref{sec:sim_MCchain} will discuss the existing and well-established
IceCube Monte Carlo (MC) chain, which has been adopted for PINGU simulations.
Here, individual neutrino events are generated and their output of Cherenkov
photons is modelled. After propagating the photons through the ice, the
resulting hit pattern is processed through the standard reconstruction and event
selection specified in Secs.~\ref{sec:EvtReco} and \ref{sec:EvtSel},
respectively. The outcome of this event-by-event MC, \ie the effective
areas, reconstruction resolutions, and particle identification efficiencies for
all neutrino flavours, are then used as input for the second part of the
detector simulation.

The Parametric PINGU Analysis, \papa in short, was written specifically for the
rapid analysis of PINGU's neutrino mass hierarchy sensitivity including a
variety of systematic parameters. Since propagating these through the full
MC chain would be way too time-consuming, an effective detector
simulation was implemented that, instead of operating on individual events,
generates the expected event distributions directly, based on the detector
performance retrieved from the MC data. \papa will be described in
detail in Sec.~\ref{sec:papa}.

%==============================================================================
\section{The IceCube/PINGU Simulation Chain}
\label{sec:sim_MCchain}
%==============================================================================

\subsection{Event Generation}
\label{sec:MC_genie}

The first step in the MC chain is to model the interaction of an incoming
neutrino with a target nucleus and the resulting final state, the so-called
event generation. In the dedicated PINGU MC, this is carried out using the
GENIE (Generates Events for Neutrino Interaction Experiments) \cite{GENIE}
software package. This is already the first modification of the standard
IceCube MC chain, where NuGen \cite{NuGen}, an IceCube-specific neutrino
generator, is the default. Yet NuGen is laid out for high-energy neutrino
events where only deep inelastic scattering has to be considered as an
interaction process. In PINGU, however, the low GeV energy range is carrying the
interesting oscillation signal, and here the complex interplay between
quasi-elastic and deep-inelastic scattering as well as resonant processes have
to be taken into account (see Sec.~\ref{sec:Xsecs}). Since GENIE puts much
effort into modelling especially this energy range with great care and
validating it against experimental results, it is the natural choice for
generating PINGU events.

GENIE starts off with an isotropic flux of neutrinos of a given flavour
following a user-defined power-law distribution in energy (usually $\propto
E^{-1}$ or $E^{-2}$ for PINGU MC \cite{PINGU_MC}) on the surface of a
cylindrical generation volume well encompassing the full IceCube detector. Any
generated neutrino passing through the interaction volume, which is fully inside
the generation volume but still contains the detector as a whole, is forced to
interact inside this volume. The interaction type is chosen randomly from the
ones that are allowed and the event is assigned a weight $\mathcal{W}_i$
proportional to the particular interaction probability, taking into account the
generated energy spectrum. This weighting strategy makes it possible to
re-weight the generated events to any desired incoming flux $\Phi(E, \theta)$
later on. Then the actual weight is simply given by
\begin{equation}
 w_i = \frac{\Phi(E_i, \theta_i)\,\mathcal{W}_i}{N_\mathrm{evts}} \quad,
 \label{eqn:reweight}
\end{equation}
where $N_\mathrm{evts}$ is the total number of simulated events.

After the interaction mechanism has been determined, the interaction itself is
modelled in detail and all involved particles, from the initial neutrino and
nucleus over possible intermediate states to the final (meta-)stable particles
like pions or muons, are stored inside an \texttt{I3MCTree} object for further
processing. The reference to a tree comes from the fact that this object has
the structure of a multiply nested list, where every particle is the root of a
sub-tree (or branch) holding the particles created in its decay. The particles
are characterised by their identities, positions, four-momenta, and state (such
as 'initial', 'intermediate', or 'final'). 
Additional GENIE-specific information such as the number of generated events,
$N_\mathrm{evts}$, the size of the interaction volume, and others, are kept as
an \texttt{I3MCWeightDict} object.

\subsection{Particle Propagation}
\label{sec:MC_clsim}

The \texttt{I3MCTree} generated by GENIE is handed off to the mmc module
\cite{mmc}, which propagates the final state particles in the tree as well as
possible secondaries created in their decay further through the ice until they
have deposited all their energy. 

\subsection{Detector Response}
\label{sec:MC_detector}

%==============================================================================
\section{The PaPA Code}
\label{sec:papa}
%==============================================================================

\subsection{Idea}
\label{sec:sim_idea}


\subsection{Implementation}
\label{sec:papa_code}


\subsection{Systematic Parameters}
\label{sec:systematics}
