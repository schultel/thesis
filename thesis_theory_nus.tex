%==============================================================================
\chapter{Theory}
\label{sec:theory}
%==============================================================================

In this chapter, the theoretical background of this thesis will be discussed
in detail. This covers both physics as well as the techniques used for data
analysis. In the physics part, we will concentrate on neutrinos and their
properties and interactions, with special emphasis on neutrino oscillations.
The analysis part will describe the Fisher Information Matrix as a tool to
efficiently characterise a multi-dimensional likelihood landscape.

%==============================================================================
\section{Neutrinos}
\label{sec:nus}
%==============================================================================


\subsection{Neutrinos in the Standard Model}
\label{sec:NusInSM}

In the Standard Model of Particle Physics, or just Standard Model, the current
theories of the electroweak and strong interactions are combined. It is a
quantum field theory of the fundamental interactions and particles relevant on
the scales that are accessible for particle physics experiments.

%TODO: Chart of SM particles - including Higgs!
The particles can be divided in two classes: fermions with an intrinsic spin of
1/2 make up everything that is usually called ``matter'', and exchange bosons
with integer (1 in most cases) spin that convey the interactions and couple to
the respective charge. Formally, the bosons are the generators of the gauge
symmetry group of the particular interaction.

This means that the strong force, which obeys a \emph{SU(3)} symmetry, has eight
generators that are represented by eight gluons $g$. The gluons couple to the
strong charge which is usually referred to as ``colour''. Since it has the
largest coupling constant, the strong interaction is dominant whenever a
colour charge is present. However colour is ``confined'', i.e. free particles
must not have a net colour. This means that any coloured particles have to be
bound inside a compound object at all times. Also, the range of the strong
interaction is limited to about the size of a nucleus since the gluons are
coloured themselves and hence self-coupling.

About two orders of magnitude weaker is the electromagnetic interaction.
According to its \emph{U(1)} symmetry, it has only one exchange boson, the
photon $\gamma$, coupling to the electrical charge. It is massless and
electrically neutral, hence the electromagnetic interaction is not restricted
in range. This and the fact that there is no confinement on the electrical
charge mean that on macroscopic scales electromagnetic phenomena are dominant.

At low energies, the effective coupling constant of the weak interaction is
another three orders below the electromagnetic one. From its \emph{SU(2)}
symmetry originate three exchange bosons, $W^\pm$ and $Z^0$ with masses of
about 90\,MeV limiting its range to the subatomic scale. However with increasing
energy, the mass of the gauge bosons becomes more and more negligible and the
effective coupling rises. Above the electroweak unification at about 100\,GeV,
the weak and electromagnetic interactions can be described by one unified
theory, whose existence is also hinted to by the fact that the weak gauge
bosons are electrically charged. 

\subsection{Neutrino Mass Hierarchy}
\label{sec:NMH}


\subsection{Neutrino Interactions}
\label{sec:NuInt}


\subsection{Atmospheric Neutrinos}
\label{sec:AtmNus}