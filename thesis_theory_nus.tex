%==============================================================================
\chapter{Theory}
\label{sec:theory}
%==============================================================================

In this chapter, the theoretical background of this thesis will be discussed
in detail. This covers both physics as well as the techniques used for data
analysis. In the physics part, we will concentrate on neutrinos and their
properties and interactions, with special emphasis on neutrino oscillations.
The analysis part will describe the Fisher Information Matrix as a tool to
efficiently characterise a multi-dimensional likelihood landscape.

%==============================================================================
\section{Neutrinos}
\label{sec:nus}
%==============================================================================


\subsection{Neutrinos in the Standard Model}
\label{sec:NusInSM}

In the Standard Model of Particle Physics, or just Standard Model, the current
theories of the electroweak and strong interactions are combined
\cite{SMGlashow, SMWeinberg, SMSalam, SMtHooft}, for an overview see e.g.\
\cite{SMtextbook}. It is a quantum field theory of the fundamental interactions
and particles relevant on the scales that are accessible for particle physics
experiments.

%TODO: Chart of SM particles - including Higgs!
The particles can be divided in two classes: fermions with an intrinsic spin of
1/2 make up everything that is usually called ``matter'', and exchange bosons
with integer (1 in most cases) spin that convey the interactions and couple to
the respective charge. Formally, the bosons are the generators of the gauge
symmetry group of the particular interaction.

This means that the strong force, which obeys a \emph{SU(3)} symmetry, has eight
generators that are represented by eight gluons $g$. The gluons couple to the
strong charge which is usually referred to as ``colour''. Since it has the
largest coupling constant, the strong interaction is dominant whenever a
colour charge is present. However colour is ``confined'', i.e. free particles
must not have a net colour. This means that any coloured particles have to be
bound inside a compound object at all times. Also, the range of the strong
interaction is limited to about the size of a nucleus since the gluons are
coloured themselves and hence self-coupling.

About two orders of magnitude weaker is the electromagnetic interaction.
According to its \emph{U(1)} symmetry, it has only one exchange boson, the
photon $\gamma$, coupling to the electrical charge. It is massless and
electrically neutral, hence the electromagnetic interaction is not restricted
in range. This and the fact that there is no confinement on the electrical
charge mean that on macroscopic scales electromagnetic phenomena are dominant.

At low energies, the effective coupling constant of the weak interaction is
another three orders below the electromagnetic one. From its \emph{SU(2)}
symmetry originate three exchange bosons, $W^\pm$ and $Z^0$ with masses of
about 90\,MeV limiting its range to the subatomic scale. However with increasing
energy, the mass of the gauge bosons becomes more and more negligible and the
effective coupling rises. Above the electroweak unification at about 100\,GeV,
the weak and electromagnetic interactions can be described by one unified
theory, whose existence is also hinted to by the fact that the weak gauge
bosons are electrically charged.

Their masses arise from another spontaneously
broken local \emph{SU(2)}$\times$\emph{U(1)} symmetry of the so-called
Higgs\footnote{After Peter Higgs, who, together with others, laid the
foundations of this theory in the 1960's \cite{Higgs, BroutEnglert}.} field.
After breaking, the generators of the \emph{SU(2)} part mix with the weak
bosons, giving them mass, while the generator of the remaining \emph{U(1)} can
be observed as the only scalar gauge boson, the Higgs boson. The Higgs boson
was the last fundamental particle of the standard model to be detected, its
discovery was claimed by the ATLAS and CMS collaborations in 2012
\cite{AtlasHiggs, CMSHiggs}.

The other group of fundamental particles are the fermions (and their
corresponding antiparticles). They can be divided again into two subclasses: the
six quarks $u,\ d,\ c,\ s,\ t$, and $b$, which obey all forces and---being
coloured---are confined, so that no free quarks can be found in nature. Bound
quarks are making up baryons, like protons and neutrons, consisting of three
quarks, and unstable mesons, which consist of a quark and an antiquark, like
pions. Baryons and mesons, together called hadrons, are the only free particles
participating in the strong interaction, since they contain coloured quarks,
although not being coloured themselves.

The second subclass are the leptons, the three charged leptons $e$, $\mu$, and
$\tau$, as well as the corresponding (neutral) neutrinos \nue, \numu, and
\nutau. The charged leptons interact predominantly electromagnetically, most
prominently electrons are bound to nuclei via electrical attraction. However
the decay of $\mu$ and $\tau$ is---like every flavour-changing process---a weak
interaction. The electron as the lightest charged lepton has to be stable due
to conservation of energy and charge.

Since neutrinos are neither coloured nor electrically charged, they only
interact weakly. This means that they are very hard to detect directly. In
fact, their existence had already been suggested in 1930 by Wolfgang Pauli as a
solution for the problem of missing energy in radioactive $\beta$ decays
\cite{PauliBeta}. However the first direct detection of (electron) neutrinos,
\nue, from a nuclear reactor was achieved only in 1956 in the so-called
Cowan-Reines experiment \cite{CowanReines}. The existence of a second neutrino,
the muon neutrino \numu, was established few years later in 1962 from the study
of charged pion decays \cite{NuMuDiscovery}. The third neutrino, the \nutau,
was finally discovered by the DONUT experiment in the decay of $D_S$ mesons into
\nutaubar and $\tau$, which again decay into \nutau and other leptons
\cite{DONUT}.

\subsection{Neutrino Mass Hierarchy}
\label{sec:NMH}


\subsection{Neutrino Interactions}
\label{sec:NuInt}


\subsection{Atmospheric Neutrinos}
\label{sec:AtmNus}