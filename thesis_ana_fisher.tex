%==============================================================================
\chapter{Analysis}
\label{sec:ana}
%==============================================================================

Now that the tools and theoretical foundations needed for the simulation of an
PINGU physics run have been presented and described, we can move on to the
actual calculation of PINGU's physics capabilities, in particular its
sensitivity for the neutrino mass hierarchy. This will be done using the Fisher
information matrix, a tool to quickly evaluate the covariance matrix of an
experiment where a large number of systematic uncertainties has to be taken
into account, which will be introduced in Sec.~\ref{sec:fisher}.

Thereafter, the simulation input that was used for the generation of the
results is documented (Sec.~\ref{sec:sim_input}) before finally the results
themselves can be presented and discussed in Sec.~\ref{sec:results_baseline}.
Looking even more into the future, in Sec.~\ref{sec:om_effects} we will estimate
the possible benefits of instrumenting PINGU with the next generation optical
modules described in Sec.~\ref{sec:Gen2DOM}. Finally, we will make use of the
Fisher matrix's ability to easily combine the results of different experiments
measuring the same physical effect to evaluate the benefits gained from the
joint analysis of PINGU and other neutrino oscillation experiments, in
particular JUNO \cite{JUNO} (Sec.~\ref{sec:comb}).


%==============================================================================
\section{Fisher Information Matrix}
\label{sec:fisher}
%==============================================================================

The Fisher information matrix, or just Fisher matrix in the following, provides
a way to estimate the full covariance matrix of an experiment and therefore the
accuracy of its intended measurement in advance. It has been widely used,
especially in cosmology \cite{Fisher_first}, but also in neutrino
astrophysics \cite{MarekDiffuse}. For a detailed discussion see \eg
\cite{Fisher_first, DETF, DETF2}.

The experiment that is to be modelled is characterised by two sets of variables:
\begin{description}
 \item[Observables] are the variables $f_n$ that will actually be measured by
  the experiment. In the case of PINGU, these are the expected event counts 
  binned in energy and zenith angle. Since the partial derivatives of the
  observables \wrt the parameters enter the calculation of the Fisher matrix,
  their dependence on the parameters has to be known either analytically or (as
  in our case) from simulations, such that these derivatives can be determined
  in an analytical or numerical way.
 \item[Parameters] are the variables $p_i$ that will be extracted from the
  measurement (\ie from the observables). These are the physical parameters
  which are the actual target of the experiment as well as nuisance parameters
  required to account for systematic uncertainties.
\end{description}
Then the Fisher matrix is defined as:
\begin{equation}
 \mathcal{F}_{ij} = \sum_n \frac{1}{\sigma_n^2} \frac{\partial f_n}{\partial
p_i} \frac{\partial f_n}
 {\partial p_j}\bigg|_\mathrm{fid.\,model}
 \label{eqn:fisher_def}
\end{equation}
Here the $\sigma_n$ denote the errors on the measurement of the observables.
Since in this analysis, these are the expected numbers of events in the bins $n$
of the analysis histograms (cf.\ Sec.~\ref{sec:papa_code}), $f_n = N_n$, we
can apply Poissonian statistics where the errors are simply given by the square
root of the number of events:
\begin{equation}
 \sigma_n = \sqrt{f_n} = \sqrt{N_n}
\end{equation}
The derivatives in (\ref{eqn:fisher_def}) are evaluated at the set of ''true''
values of $p_i$ that are used as an input for the simulation. This set of
parameters $p_i$ is called the \emph{fiducial model} and is chosen with the best
existing knowledge.

\subsection{Properties}

Once the Fisher matrix has been constructed, the covariance
matrix of the experiment is obtained by inverting the Fisher matrix. Then the
full errors of the parameters $\sigma_i^{\rm full}$ and their correlations can
be read from the covariance matrix $\mathcal{S}$:
\begin{equation}
 \sigma_i^\mathrm{full} = \sqrt{\mathcal{S}_{ii}}
                        = \sqrt{(\mathcal{F}^{-1})_{ii}} \label{eqn:sigma_full}
\end{equation}
Also the purely statistical error of a given parameter $\sigma_i^\mathrm{stat}$
(\ie the error that is obtained assuming all other parameters are fixed) can be
calculated from the corresponding diagonal element of the Fisher matrix:
\begin{equation}
 \sigma_i^\mathrm{stat} = \sqrt{(\mathcal{F}_{ii})^{-1}} \label{eqn:sigma_stat}
\end{equation}

External constraints on parameters, so-called \emph{priors}, can easily be
incorporated as well. If parameter $i$ has an prior $\sigma_i^\mathrm{ext}$, it
can simply be added to the corresponding diagonal element of the Fisher matrix:
\begin{equation}
 \mathcal{F}_{ii}^\mathrm{prior} = \mathcal{F}_{ii} +
     \left(\frac{1}{\sigma_i^\mathrm{ext}}\right)^2
\end{equation}
To fix one parameter completely, the corresponding row and column are removed
from the Fisher matrix before inversion.

The Fisher matrix allows not only to read off the full uncertainty of any
parameter and all its correlations with other parameters, but also decompose the
full error into the purely statistical part (\ref{eqn:sigma_stat}) and the
contribution arising from the uncertainties of the other parameters, denoted as
\emph{systematic} error $\sigma_i^\mathrm{syst}$ below:
\begin{equation}
 \sigma^\mathrm{syst}_i = \sqrt{(\sigma^\mathrm{full}_i)^2 -
(\sigma^\mathrm{stat}_i)^2}\quad,
 \label{eqn:syst_error}
\end{equation}
where $\sigma^\mathrm{full}_i$ is calculated without any possible priors on
parameter $i$. The correlation coefficient between two parameters $i$ and $j$
can be calculated as:
\begin{equation}
 c_{ij} =
\frac{(\mathcal{F}^{-1})_{ij}}{\sigma_i^\mathrm{full}\sigma_j^\mathrm{full}}
 \label{eqn:corr_coeff}
\end{equation}

Another important property of the Fisher matrix is its additivity. As one can
see from (\ref{eqn:fisher_def}), where the total Fisher matrix is given by 
summing the contributions from all observables, one can always add a new
measurement to the ones already accounted for just by adding the respective
Fisher matrices. If there are parameters that only appear in one of the
measurements---\eg the relative atmospheric neutrino flux scale
$r_{\mathrm{flux},\,\nue-\numu}$ that has to be considered in PINGU, but not in
JUNO as the latter relies on reactor neutrinos---the matrix of the experiment
not yet having said parameter is expanded by a corresponding row and column
filled with zeroes: As the measurement does not depend on the new parameter, the
uncertainty on its value as extracted from the measurement
(\ref{eqn:sigma_stat}) is $1/\sqrt{0} = \infty$. The full covariance matrix of
the combination of the two experiments is then simply given by the inverse of
the added Fisher matrices.

\subsection{Prerequisites}


\subsection{The Hierarchy Parameter}
\label{sec:fisher_hierarchy}

\begin{equation}
 \Delta\chi^2 = \frac{N_\mathrm{NH}-N_\mathrm{IH}}{\sqrt{N_\mathrm{NH}}}
\end{equation}
with the significance
\begin{equation}
 \mathcal{S}_{\chi^2} = \sqrt{\sum_\mathrm{bins} \left(\Delta\chi^2\right)^2}
 = \sqrt{\sum_\mathrm{bins}
     \frac{(N_\mathrm{NH}-N_\mathrm{IH})^2}{N_\mathrm{NH}} }\quad.
\end{equation}
In this case, from the Fisher Matrix we get the identical significance of
\begin{equation}
 \mathcal{S}_\mathrm{Fisher} = \frac{1}{\sigma_h} = \frac{1}{\sqrt{\sigma_h^2}}
 = \frac{1}{\sqrt{(\mathcal{F})^{-1}_{hh}}}
 = \sqrt{\mathcal{F}_{hh}}
\end{equation}
with
\begin{equation}
 \mathcal{F}_{hh} = \sum_\mathrm{bins} \frac{1}{\sigma_b^2}
   \frac{\partial N}{\partial h} \frac{\partial N}{\partial h}
  = \sum_\mathrm{bins} \frac{1}{N_\mathrm{NH}}(N_\mathrm{NH}-N_\mathrm{IH})^2
  \quad.
\end{equation}

